\section{Results}
% Structure with subsections
% Use of figures / tables (max. 5)
% General description of the results
\subsection{Differentially Expressed Genes and Network creation}
\begin{itemize}
	\item 908 differentially expressed genes with chosen thresholds section \ref{sec:methods-deg}
	\item around half are upregulated, around half downregulated (Upregulated: 495, Downregulated: 413)
	\item String was able to query 828 of them, using alternative ID/Query methods did not change this
	\item after querying additional genes that are known to be related to other eds types: resulting network with 847 nodes and 6129 edges
	\item position of known eds genes in network is on average more central (checked degree, clustering coefficient, betweenness centrality and closeness centrality)
\end{itemize}

\subsection{Enrichment analysis and clustering}

\subsubsection{MCODE}

\begin{itemize}
	\item 3 clusters with more than 15 nodes (first with 66 nodes \& 1953 connections, second with 44 nodes \& 686 connections, third with 16 nodes \& 114 connections)
	\item first and largest cluster upregulated only
	\item second one as well
	\item 3rd MCODE cluster, shown in figure \ref{fig:mcode3}, is mostly upgregulated with 2 downregulated nodes, some not relevantly differentially expressed, also only 9/16 are not known eds genes, 8 are
	\item the eds genes are all below the threshold of $|\text{log2FoldChange}| > 0.5$, 1 of them is 1 of the two downregulated genes
	\item quite interesting to see genes closely related to other eds genes, COL21A1 is also strongly upregulated ($\text{log2FoldChange} > 2$)
	\item all known EDS genes with ADAMTS2 as exception have a high Closeness Centrality
	\item regarding the enrichment of this cluster: not surprising to see extracellular matrix in two GO-Terms when testing for overrepresentation of molecular functions
	\item first go term: GO:0005201 - The action of a molecule that contributes to the structural integrity of the extracellular matrix. overall 13 of the 16 genes in this GO-Term, 6 are known eds genes, the other 6 differentially expressed (COL10A1, COL15A1, PCOLCE, COL5A3, COL18A1, COL21A1, COL27A1)
	\item second go term: GO:0030020 - A constituent of the extracellular matrix that enables the matrix to resist longitudinal stress, subtype of the first. Same EDS genes and same genes as in first term except PCOLCE, PCOLE is also the only downregulated gene [TODO: interprete]
\end{itemize}

\begin{figure}[htb!]
	\centering
	\caption*{\textbf{MCODE cluster 3}}
	\includegraphics[width=0.6\textwidth]{fig/MCODE-cluster3.png}
	\caption[MCODE cluster 3]{\centering The MCODE cluster containing many genes known to cause other types of EDS. [TODO: add legend for shape and colour]}
	\label{fig:mcode3}
\end{figure}

\subsubsection{Community Clustering}

\begin{itemize}
	\item 6 clusters with more than 15 nodes
	\item 3 very small loosely connected clusters (18 nodes \& 17 connections, 29 nodes \& 32 connections, 29 nodes \& 29 connections), two medium sized highly interconnected (76 nodes \& 1000 conections and 105 nodes \& 2330 connections) and one very large cluster (363 nodes \& 1661 connections)
	\item especially medium sized clusters highly interconnected
	\item biggest one mix of up-regulated and down-regulated genes, contains all 21 genes known to cause other EDS types
	\item both medium sized clusters mostly upregulated $\rightarrow$ interesting!
	\item smallest cluster with 18 nodes to small for meaningful enrichment analysis regarding processes
\end{itemize}