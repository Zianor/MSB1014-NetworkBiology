\section{Methods}
% Structure with subsections
% Methodology aligns to research question (appropriate resources/methods used)
% General description of the methodology
% Enough details to ensure reproducibility of results

\subsection{Analysis of Differentially Expressed Genes and Network Creation}\label{sec:methods-deg}
% DEG
\begin{itemize}
	\item data accessible at NCBI GEO database with the accession number GSE218012 \cite{Ritelli2020} 
	\item analysis with DeSeq2 in R based on the analysis exported from GEO2R \cite{DESeq2} to identily up-regulated and down-regulated genes
	\item $|\text{log2FoldChange}| > 0.5$, $\text{pValue} <0.05$, pValue adjusted with Benjamini and Hochberg False Discovery Rate
\end{itemize}

% Network creation
\begin{itemize}
	\item query differentially expressed genes from string db \cite{StringDB} (confidence cut off 0.4)
	\item query eds genes related to other eds types additionally: 21 genes retrieved from Disease Ontology with Disease Ontology ID 13359 \cite{DO}, queried from string
	\item load data about differential expression, for EDS genes and differentially expressed genes
\end{itemize}

\subsection{Enrichment analysis and clustering}
\begin{itemize}
	\item exploratory enrichment analysis on whole network
	\item large network, thus cluster before to get better insights for specific parts
	\item two different cluster methods used with different resulting cluster structure
\end{itemize}
\subsubsection{MCODE}
\begin{itemize}
	\item MCODE finds "densely connected regions in large protein-protein interaction networks that may represent molecular complexes" \cite{mcode}
	\item results in smaller clusters
	\item suited to analyse molecular function
	\item do we see clusters with other molecular functions than expected
	\item are genes clustered together with other eds genes
	\item are clusters generelly mostly upregulated or downregulated
\end{itemize}
\subsubsection{Community Clustering}
\begin{itemize}
	\item small cluster not helpful for biological processes and pathways
	\item therefore use second clustering method
	\item community clustering with GLay, "more suitable for functional interpretation" \cite{GLay}
\end{itemize}