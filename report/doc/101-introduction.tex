\section{Introduction}

% Funnel shape (start broad, become more specific, leading to the research question)
% General description of the background and gaps that will be filled by the research
% Clear research question

Ehlers-Danlos syndromes are a group of heritable connective tissue disorders that can be classified into multiple subtypes. The current classification describes 14 subtypes \cite{classification2017, Malfait2020}, with hypermobile EDS (hEDS) being the most common form. It is also the only subtype with an unknown molecular cause, leading to a diagnosis based on clinical presentation. Identifying the molecular cause is crucial to improving the diagnosis process, understanding the disease and finding potential treatment options \cite{Ritelli2020}.

Although many studies investigated several genes, no clear molecular cause with a connection to connective tissue has been established yet \cite{Caliogna2021}. While an ongoing study aims to find the genetic cause of hEDS by analysing the genes of around 1000 affected individuals, results are not expected before 2025 \cite{HEDGE}. Until then, utilising already collected data is essential to understand more about hEDS, although a diagnosis criteria changed in 2017 results in data from before the change not being usable anymore \cite{Gensemer2021, Ritelli2022}.
%
%
%\begin{itemize}
%	\item we know the clinical representation and how other eds types work (kind of)
%	\item TODO: include information of what is affected, extracellular matrix, collagen, connective tissue
%\end{itemize}

This project aims to investigate the molecular cause of hypermobile EDS by studying the influence of differentially expressed genes in hEDS patients. It mainly tries to find molecular functions and biological processes affected by differentially expressed genes that are similar to the ones affected by the molecular cause of other EDS types. This analysis eventually aims to find genes differentially expressed in hEDS that are candidates for being the molecular cause of hEDS and to relate the findings to existing research.