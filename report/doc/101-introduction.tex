\section{Introduction}

% Funnel shape (start broad, become more specific, leading to the research question)
% General description of the background and gaps that will be filled by the research
% Clear research question

Ehlers-Danlos syndromes are a group of heritable connective tissue disorders that can be classified into multiple subtypes. The current classification proposed in 2017 describes 13 subtypes \cite{classification2017} and was later extended by one additional type \cite{Malfait2020}. The most common type is hypermobile EDS (hEDS) which is also the only subtype with unknown molecular cause, leading to a diagnosis based on clinical presentation. Identifying the molecular cause is crucial to improve the diagnosis process, the understanding of the disease and to find potential treatment options \cite{Ritelli2020}.

Although many studies investigated several genes, no clear molecular cause with a connection to con- nective tissue has been established yet \cite{Caliogna2021}. While an ongoing study aims to find the genetic cause of hEDS by analysing the genes of around 1000 affected individuals, results are not expected before 2025 \cite{HEDGE}. Until then, it is essential to utilise the already collected data to understand more about hEDS. One problem with this is that the diagnosis criteria changed with the new subtype classification in 2017, causing individuals diagnosed with hEDS before this change to not fulfilling the criteria anymore and being diagnosed with Hypermobility Spectrum Disorder (HSD) \cite{Gensemer2021}. However, the clinical representation of those two diagnoses overlaps, and the terms are often used interchangeably. Both diagnoses are also often grouped as hEDS/HSD because it is currently not clear whether they are different to each other or not \cite{Gensemer2021,Ritelli2022}.


\begin{itemize}
	\item we know the clinical representation and how other eds types work (kind of)
	\item TODO: include information of what is affected, extracellular matrix, collagen, connective tissue
\end{itemize}

This project aims to investigate the molecular cause of hypermobile EDS by studying the influence of differentially expressed genes in hEDS patients.
\begin{itemize}
	\item which biological processes and pathways are affected
	\item similarities to pathways/processes/function affected by other EDS genes
	\item TODO: what do we want to do in the big picture: find candidate genes? is this to ambitious?
\end{itemize}